\documentclass[english,11pt]{article}

\usepackage[hmargin=0.9in, vmargin=0.75in]{geometry}
\usepackage[latin9]{inputenc}
\usepackage{babel}
\usepackage{subfig}
\usepackage{graphicx}
\usepackage{caption}
\usepackage[colorlinks=true,citecolor=blue,urlcolor=blue]{hyperref}
\usepackage{amsmath}
\usepackage{amssymb,amsfonts}
\usepackage{color}
\usepackage[numbers]{natbib}
\let\cite=\citet

\usepackage{verbatim}
\usepackage{listings}

\definecolor{mygreen}{rgb}{0,0.6,0}
\definecolor{mygray}{rgb}{0.5,0.5,0.5}
\definecolor{mymauve}{rgb}{0.58,0,0.82}

\lstset{ %
  backgroundcolor=\color{white},   % choose the background color
  basicstyle=\footnotesize,        % size of fonts used for the code
  breaklines=true,                 % automatic line breaking only at whitespace
  captionpos=b,                    % sets the caption-position to bottom
  commentstyle=\color{mygreen},    % comment style
  escapeinside={\%*}{*)},          % if you want to add LaTeX within your code
  keywordstyle=\color{blue},       % keyword style
  stringstyle=\color{mymauve},     % string literal style
  }

%=============================================================================
%                              Macros
%=============================================================================
\newcommand{\code}[1]{{\bf \large \color{blue}\lstinline{#1}}}

%%---------------------------------------------------%%
%%---------------------- TITLE ----------------------%%
%%---------------------------------------------------%%


\begin{document}

\title{TwoPhaseFlow}
\date{}
\maketitle
%\printindex

\section{How to set up a problem}

Create a python file with an object called \code{myTpFlowProblem} from
the class \code{TwoPhaseFlowProblem}. Here is an example. 
\vspace{2ex}

\begin{lstlisting}[language=python, frame=single]
  import proteus.TwoPhaseFlow.TwoPhaseFlowProblem as TpFlow
  myTpFlowProblem = TpFlow.TwoPhaseFlowProblem
                               (ns_model=0,
                                nd=2,
                                cfl=0.33,
                                outputStepping=outputStepping,
                                structured=structured,
                                he=he,
                                nnx=nnx,
                                nny=nny,
                                nnz=None,
                                domain=domain,
                                initialConditions=initialConditions,
                                boundaryConditions=boundaryConditions,
                                useSuperlu=False)
\end{lstlisting}

\subsection{The TwoPhaseFlowProblem class} 
This is the main class. It serves as a container for different information 
related to the problem, physical parameters and numerical methods.

\subsubsection{Input parameters}
\begin{itemize}
\item {\bf ns\_model} (default: {\it ns\_model=0}).
  Selects the Navier-Stokes solver from
  ({\it ns\_model=0}) a coupled velocity-pressure solver and
  ({\it ns\_model=1}) a projection scheme solver. See \S\ref{sec:numerical_methods}
  for more details. 

\item {\bf nd} (default: {\it nd=2}).
  Number of physical dimensions. It must be $nd=\{2,3\}$. 

\item {\bf cfl} (default: {\it cfl=0.33}).
  Desired/target CFL.

\item {\bf outputStepping} (no default).
  Object from OutputStepping class that defines and controls the outupt frequency of the solution.
  The user must define a final time and either: 1) how often the solution is output or 2) how many
  outputs to produce. 
  The user can also define {\it dt\_init} to create the first output after the initial condition.
  By default {\it dt\_init=0.001}. 
  Here are two examples.
  
\begin{lstlisting}[language=python, frame=single]
  import proteus.TwoPhaseFlow.TwoPhaseFlowProblem as TpFlow
  final_time=1.0
  outputStepping = TpFlow.OutputStepping(final_time, dt_output=0.01)  
\end{lstlisting}  

\begin{lstlisting}[language=python, frame=single]
  import proteus.TwoPhaseFlow.TwoPhaseFlowProblem as TpFlow
  final_time=1.0
  outputStepping = TpFlow.OutputStepping(final_time, nDTout=10)
\end{lstlisting}  

\item {\bf structured} (default: {\it structured=False}).
  True or False to define if the mesh is structured or not. 

\item {\bf he} (no default).
  Characteristic size of elements. This is standard in Proteus.

\item {\bf nnx, nny, nnz} (no default).
  Number of elements in x, y and z respectively. This is needed only if {\it structured=True}.

\item {\bf domain} (no default).
  Standard domain in Proteus.

\item {\bf triangleFlag} (default: {\it triangleFlag=1}).
  Defines the orientation of the mesh elements when {\it structured=True}.

\item {\bf initialConditions} (no default).
  Dictionary of initial conditions. Here is an example.

\begin{lstlisting}[language=python, frame=single]
  initialConditions = {
    'pressure': zero(),
    'pressure_increment': zero(),
    'vel_u': zero(),
    'vel_v': zero(),
    'vel_w': zero(),
    'clsvof': clsvof_init_cond()}
\end{lstlisting}  
where Zero, etc. are user defined python classes.

\item {\bf boundaryConditions} (no default).
  Dictionary of boundary conditions. Here is an example.
  
\begin{lstlisting}[language=python, frame=single]
  boundaryConditions = {
    # DIRICHLET BCs #
    'pressure_DBC': pressure_DBC,
    'pressure_increment_DBC': pressure_increment_DBC,
    'vel_u_DBC': vel_u_DBC,
    'vel_v_DBC': vel_v_DBC,
    'vel_w_DBC': vel_w_DBC,
    'clsvof_DBC': clsvof_DBC,
    # ADVECTIVE FLUX BCs #
    'pressure_AFBC': pressure_AFBC,
    'pressure_increment_AFBC': pressure_increment_AFBC,
    'vel_u_AFBC': vel_u_AFBC,
    'vel_v_AFBC': vel_v_AFBC,
    'vel_w_AFBC': vel_w_AFBC,
    'clsvof_AFBC': clsvof_AFBC,
    # DIFFUSIVE FLUX BCs #
    'pressure_increment_DFBC': pressure_increment_DFBC,
    'vel_u_DFBC': vel_u_DFBC,
    'vel_v_DFBC': vel_v_DFBC,
    'vel_w_DFBC': vel_w_DFBC,
    'clsvof_DFBC': clsvof_DFBC}
\end{lstlisting}  
where pressure\_DBC, etc. are user defined python functions.

\item {\bf useSuperlu} (no default).
  Use or not superlu. By default we don't use superlu but we allow the user to override this.  
  Note that superlu can't be used if the null space of the matrix is non empty
  (e.g. Poisson problem with Neumann boundary conditions).
  
\end{itemize}


\section{How to run a problem}
Use Proteus' {\it parun} script by passing
\begin{lstlisting}
  --TwoPhaseFlow --path path_to_problem -f name_of_problem.py
\end{lstlisting}
along with the normal set of Proteus flags and parameters.
Note: if {\it -{}-path} is not passed then {\it name\_of\_problem.py} must be located in the
current directory. Here is an example:
\begin{lstlisting}[frame=single]
  parun --TwoPhaseFlow -f risingBubble.py -l1 -v -C `final_time=3.0 refinement=4'
\end{lstlisting}

\section{Numerical methods}\label{sec:numerical_methods}
\subsubsection*{Navier-Stokes}
\begin{itemize}
\item If {\it ns\_model=0} the Navier-Stokes equations are solved via a coupled velocity-pressure
  method with $\mathbb{P}_1-\mathbb{P}_1$ finite element spaces and SUPG stabilization as in
  \cite{bazilevs2007variational, brooks1982streamline}.
  Extra artificial viscosity is added via shock capturing, see
  \cite{bazilevs2007yzbeta,tezduyar2007supg,tezduyar2006computation,tezduyar2006stabilization}.

\item If {\it ns\_model=1} the Navier-Stokes equations are solved via the projection scheme 
  by \cite{guermond2009splitting} with $\mathbb{P}_2-\mathbb{P}_1$ finite element spaces.
  Extra artificial viscosity is added to the momentum equation following
  \cite{guermond2011entropy,cappanera2017momentum}.
  Surface tension is considered in this case via the semi-implicit method by \cite{hysing2006new}. 
  
\end{itemize}

\subsubsection*{Level-set}
We consider the phase conservative level-set method by \cite{quezada2018monolithic}. 

\section{Example}
This example considers the two rising bubble benchmark problems in \cite{hysing2009quantitative}. 

\lstinputlisting[language=python, frame=single]{example/risingBubble.py}

\vspace{2ex}
To run the first test case in parallel using 4 processors type 

\begin{lstlisting}[frame=single]
  mpiexec -np 4 parun --TwoPhaseFlow -l1 -v -f risingBubble.py -C ``test_case=1 structured=True refinement=5''
\end{lstlisting}

The results for the two benchmarks at $t=3$ are shown in figure \ref{fig:rising_bubble} 

\begin{figure}[h]
  \centering
  \subfloat[Test case 1]{\includegraphics[scale=0.5]{example/test_case1.png}} \qquad \qquad \qquad
  \subfloat[Test case 2]{\includegraphics[scale=0.5]{example/test_case2.png}} 
  \caption{Solution of rising bubble benchmarks from \cite{hysing2009quantitative} at $t=3$.
  We show the level set function and its zero contour plot.}
  \label{fig:rising_bubble}
\end{figure}

\bibliographystyle{abbrvnat}
%\bibliographystyle{unsrtnat}
%\bibliographystyle{plain}
\bibliography{References.bib}


\end{document}
